\documentclass[UTF8,8pt,a4paper]{article}
\usepackage{ctex}
\usepackage{geometry}
\usepackage{amsmath}
\usepackage{amssymb}
\usepackage{bookmark}
\usepackage{graphicx}
\usepackage{url}
\usepackage{caption}
\usepackage{hyperref}
\usepackage{physics}

\geometry{left=1.8cm,right=1.8cm,top=1.9cm,bottom=1.9cm}
\makeatletter
\def\@fnsymbol#1{\ensuremath{%
  \ifcase#1\or % 0
    \dagger\or % 1 - 改为 dagger
    *\or       % 2 - 原来的 dagger 变成 asterisk
    \ddotagger\or % 3
    \S\or      % 4
    \P\or      % 5
    \|\or      % 6 (DOUBLE VERTICAL LINE)
    \or      % 7
    \dagger\dagger\or % 8
    \ddotagger\ddotagger % 9
  \else
    \@ctrerr
  \fi
}}
\makeatother

\title{\huge\textbf{刚体力学小测解答}}
\author{TA:疏宇\thanks{School of Gifted Young, USTC, email:\url{shuyu2023@mail.ustc.edu.cn}}\quad{}师驰昊\thanks{School of Gifted Young, USTC, email:\url{1984019655@qq.com}}}
\date{December 10$^{\text{th}}$, 2025}

\begin{document}
\maketitle

\section{习题 1: 进动参考系中的角动量分析}

\subsection{题目描述}
一高速自转的杠杆陀螺稳定地进动. 在跟着陀螺一起进动的参考系里, 陀螺的角动量是否恒定? 试从受力分析的角度解释之.

\subsection{详细解答}

\subsubsection{1. 结论}
\textbf{结论:} 在跟着陀螺一起进动的参考系 (旋转参考系) 里, 陀螺的角动量矢量 $\vec{L}$ 是\textbf{恒定}的 (即大小和方向均不随时间变化).

\subsubsection{2. 运动学分析 (坐标系变换)}
设陀螺的自转角动量为 $\vec{L}$, 进动角速度为 $\vec{\Omega}$.
\begin{itemize}
    \item \textbf{在惯性系 (地面系) 中:} 陀螺受到重力产生的力矩 $\vec{\tau}$ 作用. 根据角动量定理:
    \begin{equation}
        \left( \frac{d\vec{L}}{dt} \right)_{\text{in}} = \vec{\tau}
    \end{equation}
    对于稳定进动, 重力力矩 $\vec{\tau}$ 的方向始终垂直于角动量 $\vec{L}$ 和竖直轴, 其作用恰好是改变角动量的方向而不改变其大小. 此时有动力学关系:
    \begin{equation}
        \vec{\tau} = \vec{\Omega} \times \vec{L}
    \end{equation}
    这说明在惯性系中, 角动量 $\vec{L}$ 随时间转动, 并不恒定.
    
    \item \textbf{在进动系 (旋转系) 中:} 设参考系以角速度 $\vec{\Omega}$ 绕竖直轴转动. 根据矢量导数的旋转变换公式 (Transport Theorem), 任意矢量 $\vec{A}$ 在惯性系和旋转系中的时间变化率满足:
    \begin{equation}
        \left( \frac{d\vec{A}}{dt} \right)_{\text{in}} = \left( \frac{d\vec{A}}{dt} \right)_{\text{rot}} + \vec{\Omega} \times \vec{A}
    \end{equation}
    我们将角动量矢量 $\vec{L}$ 代入上式:
    \begin{equation}
        \left( \frac{d\vec{L}}{dt} \right)_{\text{in}} = \left( \frac{d\vec{L}}{dt} \right)_{\text{rot}} + \vec{\Omega} \times \vec{L}
    \end{equation}
    结合惯性系中的动力学方程 $\left( \frac{d\vec{L}}{dt} \right)_{\text{in}} = \vec{\tau} = \vec{\Omega} \times \vec{L}$, 我们得到:
    \begin{equation}
        \vec{\Omega} \times \vec{L} = \left( \frac{d\vec{L}}{dt} \right)_{\text{rot}} + \vec{\Omega} \times \vec{L}
    \end{equation}
    由此解得:
    \begin{equation}
        \left( \frac{d\vec{L}}{dt} \right)_{\text{rot}} = \vec{0}
    \end{equation}
    这从运动学推导上证明了: 在进动参考系中, 陀螺相对于该参考系静止 (仅有绕轴自转, 轴的方向不变), 因此其角动量矢量恒定.
\end{itemize}

\subsubsection{3. 受力分析 (动力学解释)}
在非惯性系中分析问题时, 必须引入惯性力矩.
在以角速度 $\vec{\Omega}$ 旋转的参考系中, 角动量定理的形式变为:
\begin{equation}
    \left( \frac{d\vec{L}}{dt} \right)_{\text{rot}} = \vec{\tau}_{\text{real}} + \vec{\tau}_{\text{inertial}}
\end{equation}
其中:
\begin{enumerate}
    \item \textbf{真实力矩 $\vec{\tau}_{\text{real}}$:} 即重力产生的力矩 $\vec{\tau} = \vec{r}_c \times m\vec{g}$. 在稳定进动中, 我们已知 $\vec{\tau}_{\text{real}} = \vec{\Omega} \times \vec{L}$.
    \item \textbf{惯性力矩 $\vec{\tau}_{\text{inertial}}$:} 这里起主要作用的是\textbf{科里奥利力矩}. 
    对于高速自转的陀螺, 其中的质量微元 $\mathrm{d}m$ 具有相对速度 $\vec{v}' = \vec{\omega} \times \vec{r}'$. 科里奥利力为 $\mathrm{d}\vec{F}_{cor} = -2 \mathrm{d}m (\vec{\Omega} \times \vec{v}')$.
    对整个刚体积分求矩, 可以证明 (对于对称陀螺) 科里奥利力产生的合力矩为:
    \begin{equation}
        \vec{\tau}_{\text{inertial}} = - \vec{\Omega} \times \vec{L}
    \end{equation}
    这个力矩的方向与重力力矩相反, 起到了``恢复''或``平衡''的作用.
\end{enumerate}

\textbf{综合分析:}
在进动系中, 总力矩为:
\begin{equation}
    \vec{\tau}_{\text{total}} = \vec{\tau}_{\text{real}} + \vec{\tau}_{\text{inertial}} = (\vec{\Omega} \times \vec{L}) + (-\vec{\Omega} \times \vec{L}) = \vec{0}
\end{equation}
由于在进动参考系中, 陀螺所受的合外力矩 (包含惯性力矩) 为零, 根据角动量定理, 其角动量保持守恒.

\newpage

\section{习题 2: 半球碗内的小球纯滚动}

\subsection{题目描述}
半径为 $r$ 的小球在半径为 $R$ 的半球形大碗内作纯滚动. 这种运动是简谐振动吗? 若是, 求其振动周期.

\subsection{详细解答}

\subsubsection{1. 物理模型与几何关系}
设小球质量为 $m$, 它是实心均匀球体, 其绕质心的转动惯量为 $I_{\text{cm}} = \frac{2}{5}mr^2$.
设小球球心与大碗球心连线与竖直方向的夹角为 $\theta$.

\begin{itemize}
    \item \textbf{轨迹半径:} 小球质心 (球心) 绕大碗中心运动的轨迹半径为 $R' = R - r$.
    \item \textbf{线速度:} 质心的线速度大小为 $v = R' \dot{\theta} = (R - r)\dot{\theta}$.
    \item \textbf{纯滚动约束:} 设小球自转角速度为 $\omega$. 纯滚动意味着接触点瞬时速度为零, 即质心速度等于自转线速度:
    \begin{equation}
        v = \omega r \implies \omega = \frac{v}{r} = \frac{R-r}{r}\dot{\theta}
    \end{equation}
\end{itemize}

\subsubsection{2. 机械能守恒法求解}
系统在运动过程中只有重力做功 (静摩擦力不做功), 故机械能守恒. 取大碗底部 (最低点) 为重力势能零点.

\textbf{(1) 动能 $T$:}
动能包含质心平动动能和绕质心转动动能:
\begin{equation}
    \begin{aligned}
        T &= T_{\text{trans}} + T_{\text{rot}} \\
        &= \frac{1}{2} m v^2 + \frac{1}{2} I_{\text{cm}} \omega^2 \\
        &= \frac{1}{2} m \left[ (R-r)\dot{\theta} \right]^2 + \frac{1}{2} \left( \frac{2}{5}mr^2 \right) \left[ \frac{R-r}{r}\dot{\theta} \right]^2 \\
        &= \frac{1}{2} m (R-r)^2 \dot{\theta}^2 + \frac{1}{5} m (R-r)^2 \dot{\theta}^2 \\
        &= \frac{7}{10} m (R-r)^2 \dot{\theta}^2
    \end{aligned}
\end{equation}

\textbf{(2) 势能 $V$:}
当偏角为 $\theta$ 时, 质心升高的高度为 $h = (R-r)(1 - \cos\theta)$.
\begin{equation}
    V = mgh = mg(R-r)(1 - \cos\theta)
\end{equation}

\textbf{(3) 建立运动方程:}
总机械能 $E = T + V = \text{const}$. 对时间 $t$ 求导:
\begin{equation}
    \frac{dE}{dt} = \frac{d}{dt} \left[ \frac{7}{10} m (R-r)^2 \dot{\theta}^2 + mg(R-r)(1 - \cos\theta) \right] = 0
\end{equation}
执行求导运算 (注意 $\frac{d}{dt}(\dot{\theta}^2) = 2\dot{\theta}\ddot{\theta}$ 和 $\frac{d}{dt}(\cos\theta) = -\sin\theta \dot{\theta}$):
\begin{equation}
    \frac{7}{10} m (R-r)^2 \cdot (2\dot{\theta}\ddot{\theta}) + mg(R-r) \sin\theta \cdot \dot{\theta} = 0
\end{equation}
假设小球在运动 ($\dot{\theta} \neq 0$), 我们可以约去公因子 $m(R-r)\dot{\theta}$:
\begin{equation}
    \frac{7}{5} (R-r) \ddot{\theta} + g \sin\theta = 0
\end{equation}
整理得到非线性微分方程:
\begin{equation}
    \ddot{\theta} + \frac{5g}{7(R-r)} \sin\theta = 0
\end{equation}

\subsubsection{3. 简谐振动近似与周期}
\textbf{结论判定:} 上述方程 $\ddot{\theta} + \omega_0^2 \sin\theta = 0$ 严格来说描述的是\textbf{物理摆}的运动, 不是标准的简谐振动 (SHM).
但在\textbf{小角度近似} ($\theta \ll 1$) 下, 我们有 $\sin\theta \approx \theta$. 此时方程化为:
\begin{equation}
    \ddot{\theta} + \frac{5g}{7(R-r)} \theta = 0
\end{equation}
这符合简谐振动方程的标准形式 $\ddot{\theta} + \omega_0^2 \theta = 0$.

\textbf{固有角频率与周期:}
由方程可知, 系统的固有角频率平方为:
\begin{equation}
    \omega_0^2 = \frac{5g}{7(R-r)}
\end{equation}
因此, 振动周期 $T$ 为:
\begin{equation}
    T = \frac{2\pi}{\omega_0} = 2\pi \sqrt{\frac{7(R-r)}{5g}}
\end{equation}

\subsubsection{4. 结果讨论}
\begin{itemize}
    \item 只有在偏角 $\theta$ 很小的情况下, 该运动才可视为简谐振动.
    \item 纯滚动条件引入的转动惯量使得等效质量增加 (系数从 1 变为 7/5), 导致周期比无摩擦滑动的单摆周期 ($2\pi\sqrt{(R-r)/g}$) 更长.
    \item 注意公式中的 $(R-r)$, 若 $r$ 不可忽略, 必须使用轨迹半径而非大碗半径.
\end{itemize}

\end{document}